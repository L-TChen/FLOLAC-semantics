\title{Semantics of Functional Programming}
\subtitle{The Scott Model of \PCF}
\author[T.-R. Chuang]{\alert{Chuang, Tyng-Ruey} and Chen, Liang-Ting  \\
  {\texttt{[\alert{trc}|lxc]@iis.sinica.edu.tw}}}
\institute[IIS, Sinica]{Institute of Information Science, Academia Sinica}
\begin{document}
\frame{\maketitle}

\section{Scott domain model}
\begin{frame}{Denotational semantics of \PCF}
  Instead of specifying \emph{how} a \PCF{} program runs, we specify \emph{what} a
  program is, the \emph{denotation} of a program. 
  \\~\\
  To assign a denotation to a program, 
  \begin{itemize}
    \item each type $\sigma$ is interpreted as a domain $D_\sigma$;
    \item a context $\Gamma = x_1 : \sigma_1, \ldots, x_n : \sigma_n$
      is interpreted as a product $\prod_{i = 1}^n D_{\sigma_i}$ of domains;
    \item in particular, each term of type $\tau$ under the empty context
      is an element of $D_\tau$. 
  \end{itemize}
\end{frame}

\begin{frame}{Interpretation of types and contexts}
  Define the denotation of a type inductively:
  \begin{definition}
  Every type $\sigma$ in \PCF{} associates with a domain $D_\sigma$ as
  follows:
    \begin{enumerate}
      \item $D_\nat \defeq \mathbb{N}_\bot$, and
      \item $D_{\tau \to \sigma} \defeq \dspace{D_\tau}{D_{\sigma}}$.
    \end{enumerate}
  \end{definition}
  Define the denotation of a context inductively on its length:
  \begin{definition}
    For each context $\Gamma = x_1 : \sigma_1,\,
    \ldots,\, x_n : \sigma_n$, the associated domain is defined as
    \[
      D_\Gamma \defeq 1 \times D_{\sigma_1} \times D_{\sigma_2}
      \times \dots \times D_{\sigma_n}
    \]
  \end{definition}
\end{frame}

\begin{frame}{Interpretation of judgements}
  To proceed with the denotational semantics, we further define
  the denotation for each judgement inductively on its derivation
  of the following form
  \begin{itemize}
    \item Every judgement $\Gamma \vdash \M : \tau$ is interpreted as a
      \emph{continuous} function
      \[
        \sem{\Gamma \vdash \M : \tau} : D_\Gamma \to D_{\tau}.
      \] 
    \item In particular, 
      \[
        \sem{ {} \vdash \M : \tau} : 1 \to D_\tau
      \]
      is identified 
      with an element $\sem{{}\vdash \M : \tau}(*) = d$ of $D_\tau$.
  \end{itemize}
  \begin{block}{Convention}
    In the following context, $\sem{\Gamma \vdash \M : \tau}(\vec{d})$
    is written as
    \[
      \sem{\Gamma \vdash \M : \tau}\;\vec{d}.
    \]
    for any sequence $\vec{d} \in D_\Gamma$ if there is no danger of ambiguity. 
  \end{block}
\end{frame}
\begin{frame}
  \begin{description}
    \item[(var)] 
      Suppose that $\Gamma \vdash \M : \tau$ is of the form
      \[
        x_1 : \sigma_1,\;\ldots,\; x_n : \sigma_n \vdash x_i : \sigma_i
      \]
      derived by the rule (var).  It is interpreted as the projection
      from $D_\Gamma$ to its $i$-th component $D_{\sigma_i}$
      \[
        \sem{x_1 : \sigma_1,\, \ldots,\, x_n : \sigma_n
          \vdash x_i : \sigma_i}\;\vec{d}  \defeq d_i
      \]
      for $i = 1,\ldots,n$ where $\vec{d} = (*, d_1, \ldots, d_n) \in
      1 \times D_{\sigma_1}
      \times \dots \times D_{\sigma_n}$.
  \end{description}
  ~\\
  Note that the denotation of this judgement is equal to 
  \[
    \sem{x_1 : \sigma_1, \ldots, x_n:\sigma_n \vdash x_i:\sigma_i}
    = \pi_i 
  \]
  where $\pi_i \colon D_\Gamma \to D_{\sigma_i}$ is the $i$-th projection
  and thus it is a continuous function. 

\end{frame}

\begin{frame}
  \begin{description}
    \item[(abs)] Let $f \defeq \sem{\Gamma, x : \sigma \vdash \M : \tau}$
      be the continuous function from $D_\Gamma \times D_\sigma$
      to $D_\tau$.
      \[ 
        \sem{\Gamma \vdash \lambda x.\, \M : \sigma \to \tau} \defeq
        \Lambda f
      \]
      where $\Lambda f: D_\Gamma \to \dspace{D_\sigma}{D_\tau}$ is the
      \emph{curried} $f$. In other words
      \[
        \left(\sem{\Gamma \vdash \lambda x.\, \M :
            \sigma\to\tau}\;\vec{d}\right)\; d = \sem{\Gamma, x : \sigma \vdash
          \M : \tau}\; (\vec{d}, d).
      \]
      ~\\
    \item[(app)] Define
      \begin{align*}
        & \sem{\Gamma \vdash \M\;\N : \tau}\; \vec{d} \\
        \defeq{} & \mathit{ev}\left(
        \sem{\Gamma \vdash
          \M : \sigma \to \tau}\; \vec{d}, \sem{\Gamma \vdash \N :
          \sigma}\; \vec{d} \right)
      \end{align*}
      where $\mathit{ev}\colon \dspace{D_1}{D_2} \times D_1 \to D_2$ is the
      \emph{evaluation map} which maps a continuous function $f\colon D_1 \to
      D_2$ with an element $d \in D_1$ to $f(d)$.
  \end{description}
\end{frame}

\begin{frame}
  The cases for $\zero$ and $\suc\;\M$ are rather obvious:
  \\~\\
  \begin{description}
    \item[(z)] $\zero$ is a constant, so it does not matter what the context is:
      \[
        \sem{\Gamma \vdash \zero : \nat}\; \vec{d} \defeq 0
      \]
      i.e.\ a constant function. 
      \\~\\
    \item[(s)] The denotation of $\suc$ is the successor function
      \[
        \sem{\Gamma \vdash \suc\;\M : \nat}\;\vec{d} \defeq
        \left (S \circ \sem{\Gamma
          \vdash \M : \nat} \right)\;\vec{d}
      \]
      where $S\colon \Natbot \to \Natbot$ is defined by
          \[
            S(n) \defeq
            \begin{cases}
              \bot & \text{if $n = \bot$} \\
              n + 1 & \text{if $n \in \mathbb{N}$.}
            \end{cases}
          \]
\end{description}
\end{frame}

\begin{frame}
  \begin{description}
    \item[(Y)] The denotation of $\fix$ is the fixpoint operation 
      \[
        \sem{\Gamma \vdash \fix x.\, \M : \sigma}\;\vec{d} \defeq
        \dfix
        \left(
          \sem{\Gamma \vdash \lambda x.\, \M : \sigma \to \sigma}\;\vec{d}
        \right)
      \]
      where $\dfix$ is defined previously as $\dfix(f) \defeq \bigsqcup_{i \in
        \mathbb{N}} f^i(\bot)$.
    \item[(ifz)] The denotation of $\ifz$ 
      \begin{align*}
        & \sem{\Gamma \vdash \ifz(\M; \M_0; x.\,\M_1) : \tau}\;\vec{d} \\
        \defeq{} & \mathit{ifz}_\tau (n , e, f)
      \end{align*}
      where
      \begin{enumerate}
        \item $n \defeq \sem{\Gamma \vdash \M  : \nat}\;\vec{d}$, 
        \item $e \defeq \sem{\Gamma \vdash \M_0 : \tau}\;\vec{d}$, 
        \item $f \defeq \sem{\Gamma \vdash \lambda x.\, \M_1 : \nat\to\tau}\; \vec{d}$,
      \end{enumerate} and $\mathit{ifz}$ is defined by
          \[
            \mathit{ifz}(n, x, f) \defeq
            \begin{cases}
              \bot & \text{if $n = \bot$,} \\
              x   & \text{if $n = 0$,} \\
              f(n-1) & \text{otherwise.} \\
            \end{cases}
          \]
      \end{description}
\end{frame}

\begin{frame}
  \begin{theorem}
    For every judgement~$\Gamma \vdash \M:\tau$, 
    the associated function
    \[
      \sem{\Gamma \vdash \M : \tau} \colon D_\Gamma \to D_{\tau}
    \]
    is Scott continuous. 
  \end{theorem}
  \begin{proof}[Proof sketch.]
    It is not hard to see that each case of $\sem{\Gamma \vdash \M : \tau}$
    is a Scott continuous function.
  \end{proof}
\end{frame}

\begin{frame}
  \begin{example}
  Consider the denotations of the following judgements. 
    \begin{enumerate}
      \item $y : \nat \vdash y : \nat$
      \item ${}\vdash \lambda x.\, \underline{0} : \nat \to \nat$
      \item ${}\vdash \fix f.\,\lambda n.\,\ifz(n; \underline{0}; x.\, f\; x) : \nat
          \to \nat$.
    \end{enumerate}
  \end{example}
  ~\\
  \begin{enumerate}
    \item $\sem{ y : \nat \vdash y : \nat }\;d = d$
      \\~\\
    \item $\sem{{}\vdash \lambda x.\,\underline{0} : \nat \to\nat} = \Lambda f$ where
      \[
        f \defeq \sem{ x : \nat \vdash \zero : \nat} = \mathit{const}_0,
      \] i.e.\ the constant function at $0$.  
      \\~\\
      \seti
 \end{enumerate}
\end{frame}

\begin{frame}
  \begin{enumerate}
      \conti
    \item
      \begin{align*}
            & \sem{ {}\vdash \fix f.\,\lambda n.\,\ifz(n; \underline{0}; x.\, f\; x) : \nat
          \to \nat} \\
        ={} & \mu(g)
      \end{align*}
        where $g : \dspace{D_\nat}{D_\nat} \to \dspace{D_\nat}{D_\nat}$ is defined by
        \begin{align*}
           g & \defeq \sem{f : \nat \to \nat \vdash \lambda
          n.\,\ifz(n;\underline{0}; x.\,f\;x) : \nat \to \nat} \\
            & = \Lambda \sem{f : \nat \to \nat, n : \nat \vdash \ifz(n;
          \underline{0}; x.\,f\;x) : \nat}
        \end{align*}
        and
        \begin{align*}
         & \sem{f: \nat \to \nat, n : \nat \vdash
           \ifz(n; \underline{0}; x.\, f\;x) : \nat}\; (h,d) \\
         ={} & \mathit{ifz}(d, 0, h) 
        \end{align*}
    \end{enumerate}
\end{frame}

\begin{frame}
  Then, what is $\mu(g)$? Let's calculate $g(\bot)$ and $g^2(\bot)$. 
  \\~\\
  \[
    g(\bot_{D_\nat \to D_\nat})\;d = \mathit{ifz}(d, 0, \bot_{D_\nat \to D_\nat}) = 
    \begin{cases}
      \bot & \text{if $d = \bot$} \\
      0 & \text{if $d = 0$} \\
      \bot & \text{otherwise.} \\
    \end{cases}
  \]
  \[
    g(g(\bot))\;d
    = \mathit{ifz}(d, 0, g(\bot))
    = 
    \begin{cases}
      \bot & \text{if $d = \bot$} \\
      0 & \text{if $d = 0, 1$} \\
      \bot & \text{otherwise.} \\
    \end{cases}
  \] 
  By induction, we can show that 
  \[
    g^i(d) = 
    \begin{cases}
      \bot & \text{if $d = \bot$} \\
      0 & \text{if $d < i$} \\
      \bot & \text{otherwise,}
    \end{cases}
  \]
  so $\mu(g)\;d= 0$ if $d \neq \bot$ and $\mu(g)\;d = \bot$ if $d = \bot$. 
\end{frame}

\begin{frame}
  \begin{block}{Exercise}
    Consider the denotations of the following judgements.
    \begin{enumerate}
      \item $y : \nat \vdash (\lambda x.\,\underline{0})\; y : \nat$
      \item ${}\vdash \lambda n.\,\ifz(n; \underline{0}; x.\, x) : \nat \to \nat$
      \item ${}\vdash \lambda n.\,\ifz(n; \underline{1}; x.\, \underline{0}) : \nat
          \to \nat$
      \end{enumerate}
  \end{block}
\end{frame}

\section{Substitution and Compactness}

\begin{frame}{Substitution Lemma}
  \begin{lemma}
    Let $\Gamma = x_1 : \sigma_1, \ldots, x_n : \sigma_n$ be a context, and
    $\Gamma \vdash \M : \tau$ a judgement.  Then the following equation 
    \begin{align*}
      & \sem{\Delta \vdash \M[\vec{N}/\vec{x}]}\; \vec{d} \\
      ={} & \sem{\Gamma \vdash \M}
      \left(
        \sem{\Delta\vdash \N_1}\;\vec{d}, \ldots, 
        \sem{\Delta\vdash \N_n}\;\vec{d}
      \right)
    \end{align*}
    holds for any context $\Delta$ and judgements $\Delta \vdash \N_i :
    \sigma_i$ for $i = 1, \ldots, n$.
  \end{lemma}
\end{frame}

%\begin{frame}{Proof of Substitution Lemma}
%  \begin{description}
%    \item[(var)] Suppose that $\Gamma \vdash \M : \tau$ is of the form
%      \[
%        x_1 : \sigma_1, \ldots, x_n : \sigma_n \vdash x_i : \sigma_i
%      \]
%      for $i = 1, \ldots, n$. \\ 
%      Then, for each family of judgements
%      $\Delta \vdash \N_i : \sigma_i$, 
%      it follows that 
%      \begin{align*}
%          & \sem{\Delta \vdash x_i[\vec{n}/\vec{x}] : \sigma_i} \\
%        ={} & \sem{\Delta \vdash \N_i : \sigma_i} \\
%        ={} & \pi_i\left(\sem{\Delta \vdash \N_1 : \sigma_1}, \ldots,
%        \sem{\Delta\vdash \N_n : \sigma_n}\right) 
%      \end{align*}
%      where $\sem{x_1 : \sigma_1, \ldots, x_n : \sigma_n\vdash x_i:\sigma_i}
%      = \pi_i$. 
%  \end{description}
%\end{frame}
%
\begin{frame}
  \begin{corollary}[Application]
    For every judgement $\Gamma, x : \sigma \vdash \M : \tau$
    and $\Gamma \vdash \N : \sigma$, we have
    \[
      \sem{\Gamma \vdash (\lambda x.\,\M)\;\N : \tau}
      = \sem{\Gamma \vdash \M[\N/x] : \tau}.
    \]
  \end{corollary}
  Observe that
  \[
    \vec{d} = (\sem{\Gamma \vdash x_1 : \sigma_1}\;\vec{d}, \ldots,
  \sem{\Gamma \vdash x_n : \sigma_n}\;\vec{d})
  \]
  for any context $\Gamma = x_1 : \sigma_1, \ldots, x_n : \sigma_n$. Then, this
  corollary is a series of simple facts:
    \begin{align*}
          & \sem{\Gamma \vdash (\lambda x.\,\M)\; \N : \tau}\;\vec{d} \\
      ={} &
        \mathit{ev}
        \left(
          \sem{\Gamma \vdash (\lambda x.\, \M) : \sigma\to\tau}\;\vec{d},
          \sem{\Gamma \vdash \N : \sigma}\;\vec{d}
        \right) \\
        ={} & 
        \sem{\Gamma, x : \sigma \vdash \M : \tau}\;
        (\vec{d}, \sem{\Gamma \vdash \N : \sigma}\;\vec{d}) \\
        ={} &
        \sem{\Gamma \vdash \M[ \vec{x}, \N / \vec{x}, x] : \tau}\;\vec{d} \\
        ={} & 
        \sem{\Gamma \vdash \M[\N / x ] : \tau}\;\vec{d}
    \end{align*}
\end{frame}

\begin{frame}
  \begin{example}
    The denotation of 
    \[
      {}\vdash (\lambda n.\,\ifz(n; \underline{1}; x.\, x))\;\underline{1} : \nat
    \]
    and
    \[
      {}\vdash \ifz(\underline{1}; \underline{1}; x.\,x) : \nat
    \]
    are equal and calculated as follows:
    \begin{align*}
          & \sem{{}\vdash \lambda n.\,\ifz(n; \underline{1}; x.\,x)\;\underline{1}}\\
      ={} & \sem{{}\vdash \lambda n.\,\ifz(n; \underline{1}; x.\,x)}( \sem{{}\vdash
        \underline{1} : \nat}) \\
      ={} & \mathit{ifz}(1, 1, \mathit{id}) = 0
    \end{align*}
  \end{example} 
\end{frame}

\begin{frame}
  \begin{lemma}[Weakening]
    Let $\Gamma \vdash \M : \tau$ be a judgement. Then the following
    \[
      \sem{\Gamma \vdash \M : \tau}\;\vec{d}
      = \sem{\Gamma, x : \sigma \vdash \M : \tau}\;(\vec{d}, d)
    \]
    holds for any variable $x : \sigma$ not in $\Gamma$,
    $\vec{d} \in D_\Gamma$ and $d \in D_\sigma$. 
  \end{lemma}
  It follows from Substitution Lemma. (\emph{Why?})
\end{frame}

\begin{frame}
  \begin{corollary}[$\eta$-conversion]
    Let $\Gamma \vdash \M : \sigma \to \tau$ be a judgement. Then, 
    \[
      \sem{\Gamma \vdash \lambda x.\, \M\; x : \sigma\to\tau}
      = \sem{\Gamma \vdash \M : \sigma\to\tau}
    \]
    if $x$ is not a variable in $\Gamma$. 
  \end{corollary}
  For every sequence $\vec{d} \in D_\Gamma$ and $d \in D_\sigma$, we have
  \begin{align*}
    & \left( \sem{\Gamma \vdash \lambda x.\, \M\; x : \sigma \to \tau}
      \;\vec{d}\right)\; d \\
    ={} & \sem{\Gamma, x : \sigma \vdash \M\;x : \tau}(\vec{d}, d) \\
    ={} & \mathit{ev}\left(
      \sem{\Gamma, x : \sigma \vdash \M : \sigma\to\tau}(\vec{d}, d), 
      \sem{\Gamma, x : \sigma \vdash x : \sigma}(\vec{d}, d)\right)\\
    ={} & \mathit{ev}\left(
      \sem{\Gamma, x : \sigma \vdash \M : \sigma\to\tau}(\vec{d}, d), d\right)\\
    ={} & \left( \sem{\Gamma, x : \sigma \vdash \M : \sigma \to \tau}
      (\vec{d}, d)\right) d \\
    ={} & \left( \sem{\Gamma \vdash \M : \sigma \to \tau}\;\vec{d}\right)\;d. 
  \end{align*}
\end{frame}

\begin{frame}{Compactness}
  Define $\fix^i x.\, \M$ inductively for each $i \in \mathbb{N}$ by
  \begin{enumerate}
    \item $\fix^0 x.\, \M \defeq \fix x.\, x$ and
    \item $\fix^{n + 1} x.\, \M \defeq
      \M[\fix^n x.\, \M/ x]$. 
  \end{enumerate}
  ~\\
  \begin{theorem}
    For every judgement $\Gamma, x : \sigma \vdash \M : \sigma$, we have
    \[
      \sem{\Gamma \vdash \fix x.\, \M : \sigma}
      = \bigsqcup_{i \in \mathbb{N}} \sem{\Gamma \vdash \fix^i x.\, \M :
        \sigma}.
    \]
  \end{theorem}
  To show this theorem, it suffices to show the following 
  \[
    \sem{{}\vdash \fix^i x.\,\M : \sigma}
    = \sem{x : \sigma \vdash \M : \sigma}^i(\bot)
  \]
  for $i \in \mathbb{N}$. (Why?)
\end{frame}

\begin{frame}
  \begin{description}
    \item[For $n = 0$] we show that $\sem{{}\vdash \fix^0 x.\,\M : \sigma} =
      \bot_{D_\sigma} \in D_\sigma$.
      \\~\\

      By definition, $\fix^0 x.\, \M$ is equal to
     $\fix x.\, x$, so  
      \begin{align*}
        \sem{{}\vdash \fix x .\, x : \sigma} & = \mu(\id) = \bigsqcup_{i \in
          \mathbb{N}}  \id^i(\bot) \\
        &= \bigsqcup \bot = \bot
      \end{align*}
    \item[For $i = n+1$] it suffices to show that
      \[
        \sem{{}\vdash \fix^{n+1} x.\,\M : \sigma} = \sem{x : \sigma \vdash \M :
          \sigma} \left(\sem{{}\vdash \fix^n x.\,\M:\sigma}\right),
      \] so the statement follows by the induction hypothesis.
      \\~\\

      By definition, $\fix^{n+1} x.\,\M$ is equal to
      $\M[\fix^n x.\,\M/x]$, and by Substitution Lemma we have
      \begin{align*}
        \sem{{}\vdash \M[\fix^n x.\,\M/x]}
        = \sem{x : \sigma \vdash \M : \tau} \left(
          \sem{{}\vdash \fix^n x.\, \M} \right).
      \end{align*}
  \end{description}
\end{frame}


\begin{frame}
  \begin{block}{Exercise}
    Find the denotation of
    \[
      {}\vdash \fix f.\,\lambda n.\,\ifz(n; \underline{0};
      m.\, \ifz((f\; m); \underline{1}; x.\,\underline{0})) : \nat\to\nat
    \]
  \end{block}
\end{frame}

\end{document}
