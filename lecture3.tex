\title{Semantics of Functional Programming}
\subtitle{The Scott Model of \PCF}
\author[T.-R. Chuang]{Chuang, Tyng-Ruey \\
  \href{mailto@trc.iis.sinica.edu.tw}{\texttt{trc.iis.sinica.edu.tw}}}
\institute[IIS, Sinica]{Institute of Information Science, Academia Sinica}
\begin{document}
\frame{\maketitle}
\section{Scott domain model of PCF}  
\begin{frame}{Interpretation of types and contexts}
  Every type is interpreted as a domain and a context is a product of domains:
  \begin{definition}
  Every type $\sigma$ in \PCF{} associates with a domain $D_\sigma$ as
  follows:

    \begin{enumerate}
      \item $D_\nat \defeq \mathbb{N}_\bot$, and
      \item $D_{\tau \to \sigma} \defeq \dspace{D_\sigma}{D_{\tau}}$.
    \end{enumerate}
  \end{definition}
  \begin{definition}
    For each context $\Gamma = x_1 : \sigma_1, 
    \ldots, x_n : \sigma_n$, the associated domain is defined as
    \[
      D_\Gamma \defeq D_{\sigma_1} \times D_{\sigma_2}
      \times \dots \times D_{\sigma_n}
    \]
    and the associated domain of the empty context is $1 = \{*\}$. 
  \end{definition}
\end{frame}

\begin{frame}{Interpretation of simply type lambda calculus
  and natural numbers}
  Every judgement $\Gamma \vdash \M : \tau$ is interpreted as
  a Scott-continuous function
  \[
    \sem{\Gamma \vdash \M : \tau} : D_\Gamma \to D_{\tau}
  \]
  and defined inductively on the derivation of $\Gamma \vdash \M:\tau$ as
  follows. 

  \begin{enumerate}
    \item For (var), we interpret the judgement as the projection
      from the product to the component:
      \[
        \sem{x_1 : \sigma_1, \ldots, x_i : \sigma_i, ldots, x_n : \sigma_n \vdash x_i : \sigma_i} \defeq \pi 
      \]
      or pointwise
      \[
        \sem{\Gamma, x : \tau, \Delta \vdash x : \tau}(\vec{x}) = x
      \]
  \end{enumerate}
  \[ 
    \sem{\Gamma \vdash \lambda x.\, \M : \sigma \to \tau} \defeq
    \Lambda\sem{\Gamma, x : \sigma \vdash \M : \tau}
  \]
  \[
    \sem{\Gamma \vdash \M\;\N : \tau} = \mathit{ev}
    \circ \<\sem{\Gamma \vdash
      \M : \sigma \to \tau}, \sem{\Gamma \vdash \N : \sigma}\>
  \]
  \[
    \sem{\Gamma \vdash \zero : \nat}(\vec{d}) \defeq 0
  \]
  \[
    \sem{\Gamma \vdash \suc\;\M : \nat} \defeq S \circ \sem{\Gamma
      \vdash \M : \nat}
  \]
\end{frame}

\begin{frame}{Interpretation of general recursion and if-zero test}
  \[
    \sem{\Gamma \vdash \fix x.\, \M : \sigma} \defeq
    \mu \circ \Lambda\sem{\Gamma, x : \sigma \vdash \M : \sigma}
  \]
  \begin{align*}
    & \sem{\Gamma \vdash \ifz(\M; \M_0; \M_1) : \tau} \\
    \defeq{} & \mathit{ifz}_\tau \circ \<\,
    \sem{\Gamma \vdash \M  : \nat},
    \sem{\Gamma \vdash \M_0 : \tau},
    \Lambda\sem{\Gamma, x : \nat \vdash \M_1 : \tau}\,\>
  \end{align*}
  where $S$ and $\mathit{ifz}_\tau$ are defined by
  \begin{columns}
    \column{.5\textwidth}
      \[
        S(n) \defeq
        \begin{cases}
          \bot & \text{if $d = \bot$} \\
          n + 1 & \text{otherwise.}
        \end{cases}
      \]
    \column{.5\textwidth}
      \[
        \mathit{ifz}_\tau(n, x, f) \defeq
        \begin{cases}
          \bot & \text{if $n = \bot$,} \\
          x   & \text{if $n = 0$,} \\
          f(m) & \text{if $n = m +1$.} \\
        \end{cases}
      \]
  \end{columns}
\end{frame}

\begin{frame}
  In particular, every program~$\M:\tau$ associates with a function from~$1$
  to~$\sem{\tau}$, and thus determines a unique element~$\sem{\M}(*)$
  of~$\sem{\tau}$. 
  \begin{block}{Convention}
    For brevity, we write $\sem{\M}$ instead of~$\sem{{}\vdash\M : \tau}(*)$ for
    every program~$\M$ of type~$\tau$
  \end{block}
  \begin{theorem}
    For every judgement~$\Gamma \vdash \M:\tau$, 
    the associated function
    \[
      \sem{\Gamma \vdash \M : \tau} \colon \sem{\Gamma} \to \sem{\tau}
    \]
    is Scott continuous. 
  \end{theorem}
\end{frame}

\end{document}
