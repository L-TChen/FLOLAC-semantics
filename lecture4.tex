\title{Semantics of Functional Programming}
\subtitle{Lecture IV: Computational Adequacy}
\author[L.-T. Chen]{Chen, Liang-Ting\\
  \href{mailto:lxc@iis.sinica.edu.tw}{\texttt{lxc@iis.sinica.edu.tw}}}
\institute[IIS, Sinica]{Institute of Information Science, Academia Sinica}
\begin{document}
\frame{\maketitle}

\begin{frame}{Overview}
  So far we have given two kinds of semantics for \PCF{}. For a program~$\M$ of
  type~$\sigma$,
  \begin{itemize}
    \item one gives how the program~$\M$ is evaluated to a value~$\V$
      via the reduction relation~$\M \Downarrow\V$;
    \item the other defines what the denotation~$\sem{\M}$ of~$\M$ is in a
      domain~$D_\sigma$. 
    \end{itemize}
  In this lecture, we will compare these two approaches and discuss some issues
  arising from them:
  \begin{description}[Computational Adequacy]
    \item[Correctness] $\M \Downarrow \V$ implies $\sem{\M} = \sem{\V}$.
    \item[Completeness] $\sem{\M} = n$ implies $\M \Downarrow \underline{n}$
    \item[Computational adequacy]
      Both of correctness and completeness hold. 
  \end{description}

\end{frame}

\section{Correctness}

\begin{frame}{$\nat$ values always converges}
  The bottom element~$\bot$ models the divergence of computation.  A 
  value of~$\nat$ is meant to be some natural number, so it shouldn't
  diverge. 
  \\~\\
  \begin{lemma}
    For every value~$\V$ of type~$\nat$, the denotation~$\sem{\V}$ is an
    element of~$\mathbb{N}$. In particular, $\sem{\V} \neq \bot$.
  \end{lemma}
  \begin{proof}
    By structural induction on values:
    \begin{columns}[b]
      \column{.3\textwidth}
        \begin{prooftree}
          \AXC{}
          \UIC{$\zero\;\,\val$}
        \end{prooftree}
      \column{.3\textwidth}
        \begin{prooftree}
          \AXC{$\M\;\,\val$}
          \UIC{$\suc\;\M\;\,\val$}
        \end{prooftree}
      \column{.3\textwidth}
        \begin{prooftree}
          \AXC{$\M\;\,\term$}
          \UIC{$\lambda x.\, \M\;\,\val$}
        \end{prooftree}
      \end{columns}
  \end{proof}
\end{frame}

\begin{frame}
  \begin{theorem}
    For every two programs~$\M$ and~$\V$, $\M \Downarrow \V$ implies
    $\sem{\M} = \sem{\V}$. 
  \end{theorem}
  \begin{proof}[Proof sketch.]
    Prove $\sem{\M} = \sem{\V} \in \sem{\tau}$ by structural induction on the
    derivation of~$\M\Downarrow\V$. 
  \end{proof}
  ~\\
    We show the case ($\Downarrow$-suc) first and the cases
    ($\Downarrow$-zero) and ($\Downarrow$-lam) are similar and easy. 
    \\~\\
    \begin{itemize}
      \item For ($\Downarrow$-suc), we show that $\sem{\suc\;\M}
        =\sem{\suc\;\V}$ if $\sem{\M} = \sem{\V}$. By definition, we simply
        calculate its denotation directly:
        \[
          \sem{\suc\;\M} = S(\sem{\M}) = S(\sem{\V}) = \sem{\suc\;\V}
        \]
        where the middle equality follows from the induction hypothesis. 
    \end{itemize}
    ~\\
    Try to do the cases ($\Downarrow$-zero), ($\Downarrow$-lam), and
    ($\Downarrow$-$\ifz_0$).
\end{frame}

\begin{frame}
  The case ($\Downarrow$-app) is slightly complicated, as we have to address
  the binding structure using Substitution Lemma.  
  \\~\\
  \begin{itemize}
      \item For ($\Downarrow$-app), we show that $\sem{\M\;\N} = \sem{V}$
        if $\sem{\M} = \sem{\lambda x.\, \mathsf{E}}$ and $\sem{E[\N/x]} =
        \sem{V}$. We calculate the denotation as follows
        \begin{align*}
          \sem{\M\;\N} & = \mathit{ev}(\sem{\M}, \sem{\N}) \\
          & = \mathit{ev}(\sem{\lambda x.\, \mathsf{E}}, 
          \sem{\N}) \\
          & = \mathit{ev}(\sem{x : \sigma \vdash \mathsf{E} : \tau},
          \sem{\N}) \\
          & = \sem{x : \sigma \vdash \mathsf{E} : \tau}(\sem{\N})
          = \sem{\mathsf{E}[\N/x]} = \sem{\V}
        \end{align*}
        where the last but one equation follows from Substitution Lemma.
        \\~\\
      \item Complete the remaining two (interesting) cases
        ($\Downarrow$-$\ifz_1$) and ($\Downarrow$-fix).
        \emph{Hint.} Consider Substitution Lemma and
        the properties of the fixpoint operator~$\mu$.
    \end{itemize}
\end{frame}

\note{
\begin{itemize}
  \item For ($\Downarrow$-$\ifz_0$), assuming $\sem{\M}=\sem{\zero} = 0$
    and $\sem{\M_0} = \sem{V}$ we show that
    $\sem{\ifz(\M; \M_0; x.\, \M_1)} = \sem{\V}$. We calculate the
    denotation as follows
    \begin{align*}
      \sem{\ifz(\M;\M_0;x.\,\M_1)}
        & = \mathit{ifz}(\sem{\M}, \sem{\M_0}, \sem{\M_1}) \\
        & = \mathit{ifz}(0, \sem{V}, \sem{\M_1}) \\
        & = \sem{V}
    \end{align*}
    where the last equation follows from the definition of~$\mathit{ifz}$. 
  \item For ($\Downarrow$-$\ifz_1$), we show that $\sem{\ifz(\M;\M_0;
      x.\,\M_1)}$ = \sem{V} if $\sem{\M} = \sem{\suc\;\N} = S \circ \sem{\N}$
    and $\sem{\M_1[\N/x]} = \sem{\V}$.
  
  We know that~$\V$ is a value for every~$\M\Downarrow\V$ and $\sem{\V}
  = n$ for some natural number~$n \in \mathbb{N}$, so $\sem{\M}$ is a natural
  number $n + 1$. It follows that 
  \begin{align*}
    \sem{\ifz(\M; \M_0; x.\,\M_1)}
    & = \mathit{ifz}(\sem{\M}, \sem{\M_0}, \sem{x : \nat\vdash \M_1 :
      \tau}) \\
    & = \sem{x : \nat \vdash \M_1}(n) \\
    & = \sem{x : \nat \vdash \M_1}(\sem{N}) \\
    & = \sem{\M_1[\N/x]} = \sem{\V}
  \end{align*}
  where the last but one equation follows from Substitution Lemma.

  \item For ($\Downarrow$-fix), we show that $\sem{\fix x.\,\M} = \sem{\V}$ if
    $\sem{\M[\fix x.\,\M/x]} = \sem{\V}$.  Let $f \defeq \sem{x : \sigma \vdash
      \M : \sigma}$.  We
    calculate the denotation as follows
    \begin{align*}
      \sem{\fix x.\,\M}
      & = \mu f = f(\mu f) \\
      & = \sem{x : \sigma \vdash \M : \sigma}(\sem{\fix x.\, \M}) \\
      & = \sem{\M[\fix x.\, \M/x]} = \sem{V}
    \end{align*}
    where the last but one equality follows from Substitution Lemma.
\end{itemize}
}

\section{Equational reasoning}

\begin{frame}{Logical equivalence}
  \begin{definition}[Applicative approximation]
    For each type~$\sigma$, define a relation~$\precsim_\sigma$ between
    programs of~$\sigma$:
    \begin{enumerate}
      \item For the type $\nat$, define
        \[
          \M \precsim_{\alert{\nat}} \N
        \]
        if for all $n \in \mathbb{N}$, 
          $\M \Downarrow \underline{n}$ implies $\N \Downarrow \underline{n}$
      \item For every type~$\sigma \to \tau$, define
        \[
          \M \precsim_{\alert{\sigma \to\tau}} \N
        \]
          if for all program~$P$,
            $\M\;P \precsim_{\tau} \N\;P$.
    \end{enumerate}
  \end{definition}
  ~\\

    Two programs $\M$ and $\N$ of the same type~$\sigma$ are \textbf{logically
      equivalent} denoted~$\M\simeq_\sigma\N$ if $\M \precsim_\sigma \N$ and
    $\M \succsim_\sigma \N$. 

\end{frame}

\begin{frame}
  The relation $\precsim_\sigma$ is a preorder, so
  $\simeq_\sigma$ is indeed an equivalence.
  \\~\\

  \begin{proposition}
    The logical equivalence~$\simeq_\sigma$ is an equivalence relation.
  \end{proposition}
  ~\\

  A program~$\M$ can be replaced by another program~$\N$ without changing
  results if $\M \simeq_\sigma \N$.
  \\~\\
    \begin{example}
      The following two programs are logically equivalent:
      \[
        \lambda x.\, x
        \quad\text{and}\quad
        \lambda x.\, \mathtt{pred}\; (\mathtt{suc}\; x)
      \]
    \end{example}
\end{frame}

\begin{frame}{Reduction respects logical equivalence}
  Recall that from $\M \leadsto^* \M'$ and $\M' \Downarrow \V$
  it follows that $\M \Downarrow \V$ in the agreement between $\leadsto$ and
  $\Downarrow$. 
  \\~\\
  \begin{proposition}
    Let $\M$ and $\M'$ be programs of type~$\sigma$. 
    If $\M \leadsto^* \M'$, then $\M \succsim_\sigma \M'$. 
  \end{proposition}
  ~\\
  The other direction follows from the determinacy and values cannot be
  reduced further:
  \\~\\
  \begin{proposition}
    For every $\M \Downarrow \V$ and $\M \leadsto^* \M'$, 
    we have $\M' \Downarrow \V$. 
  \end{proposition}
  \begin{corollary}[Reduction entails logical equivalence]
    If $\M \leadsto^* \M'$, then $\M \simeq_\sigma \M'$. 
  \end{corollary}
\end{frame}
\note{
  \begin{proof}
    We prove it by induction on~$\sigma$. 
    \begin{description}
      \item[$\nat$] If $\M' \Downarrow \underline{n}$,
        then as we have shown in the agreement between~$\leadsto$
        and~$\Downarrow$ from $\M \leadsto \M'$ it follows $\M \Downarrow
        \underline{n}$.  Therefore, $\M \succsim_\nat \M'$. 
      \item[$\sigma\to\tau$]
        By assumption that~$\M \leadsto \M'$ and ($\leadsto$-lapp), it follows
        that $\M\; P \leadsto \M'\; P$ for every term.  By induction
        hypothesis, it follows that $\M\;P \succsim_\tau\M'\;P$ for every program of
        type~$\sigma$, so $\M \succsim_{\sigma\to\tau} \M'$.
    \end{description}
  \end{proof}
}
\begin{frame}
  However, logical equivalence goes beyond reduction.  
  Consider the following two programs of type~$\nat\to\nat\to\nat$:
    \[
      \lambda x.\, \lambda y.\, x + \alert{y}
    \]
    and 
    \[
      \lambda x.\, \lambda y.\, \alert{y} + x
    \]
    Surely the addition of natural numbers are commutative, but \emph{why?}
    By definition they are already values, so they cannot be reduced to
    each other. 
    \\~\\
    \begin{remark}
      We can show directly that these two programs are logically
      equivalent in dependent type theory. Yet, we will present an external
      approach using denotational semantics in the absense of
      the identity type. 
    \end{remark}
\end{frame}

%\begin{frame}{Observational equivalence}
%  Another proposal of equivalence between programs is given by
%  \emph{observations} or say, \emph{measurements}:
%  \begin{definition}[Observational approximation]
%    Define a relation~$\lesssim_\sigma$ for each type~$\sigma$ by
%    \[
%      \M \lesssim_\sigma \N
%      \quad\text{if}\quad
%      \forall P \in \mathsf{Prg}_{\sigma\to\nat}.\, 
%      P\;\M \mathrel{\precsim_\nat} P\;\N
%    \]
%   Two programs~$\M$ and $\N$ of type~$\sigma$ are \textbf{observationally
%     equivalent} if $\M \lesssim_\sigma \N$ and $\N \lesssim_\sigma \M$. 
%  \end{definition}
%
%\end{frame}
%
%\begin{frame}{Observational equivalence}
%  To avoid the universal quantifier, one may expect that $\sem{\M} = \sem{\N}$
%  implies $\M \Downarrow \N$ for all~$\M$, $\N$, but it is 
%  impossible because of ($\Downarrow$-lam). E.g.
%    \[
%      \sem{\lambda x.\, x}(n) = n  
%       = (n + 1) - 1 = 
%      \sem{\lambda x.\, \mathtt{pred}\;(\suc\;x)}(n)
%    \]
%  We can, however, observe programs on~$\nat$.
%  \begin{definition}[Observational approximation]
%    For every type~$\sigma$, define a relation $\lesssim_\sigma$ for
%    programs of type~$\sigma$ as 
%    \[
%      \M \lesssim_\sigma \N \quad\text{if}\quad 
%      P\; \M \precsim_\nat P\;\N.
%    \]
%    for every program~$P$ of type~$\sigma\to\nat$. 
%    Two programs~$\M$ and~$\N$ are \textbf{observationally equivalent}
%    if $\M \lesssim \N$ and $\N \lesssim \M$.
%  \end{definition}
%\end{frame}
%
%\begin{frame}{Equivalences}
%  We have introduced three different equivalences between programs:
%  \begin{enumerate}
%    \item Denotational equivalence $\sem{\M} = \sem{\N}$
%    \item Logical equivalence $\M \simeq \N$
%    \item Observational equivalence $\M \lesssim \N$ and $\N \lesssim \M$. 
%  \end{enumerate}
%  In the following, we will show that these three equivalences
%  are actually the same, using \emph{computational adequacy}. Therefore, we can
%  avoid the universal quantifier and ensure the correctness of programs by
%  calculating their denotations. 
%\end{frame}
\section{Computational Adequacy}
\begin{frame}
  In the following, we will show that for every program~$\M$ of type~$\nat$ if
  $\sem{\M} = n$ then $\M$ reduces to the numeral~$\underline{n}$.
  \\~\\
  \begin{itemize}
    \item Define a relation~$R_\sigma$ for each type~$\sigma$ between the
      domain~$\sem{\sigma} = D_\sigma$ and the collection of
      programs of type~$\sigma$:
      \[
        R_\sigma \subseteq D_\sigma \times \mathsf{Prg}_\sigma
      \]
      for every type~$\sigma$ where $\mathsf{Prg}_\sigma = \set{ \M }{{}\vdash
        \M : \sigma}$.
      \\~\\
    \item Then prove that $\sem{\M}\mathrel{R_\sigma} \M$ for every
      program~$\M$ of type~$\sigma$, and, by construction
      $\sem{\M} \mathrel{R_\nat} \M$ is equivalent to that
      $\sem{M} = n$ implies $\M \Downarrow \underline{n}$. 
  \end{itemize}
  ~\\

  With this property, we can conclude that denotational equivalence entails
  logical equivalence.\footnote{
    But, the converse may fail. 
  }
\end{frame}
\begin{frame}{Logical relation between semantics and syntax}
\begin{definition}[Logical relation]
  For every type~$\sigma$, define a relation~$\alert{R_\sigma} \subseteq
  D_\sigma \times \mathsf{Prg}_\sigma$ inductively as follows:
  \begin{itemize}
    \item $d \alert{\mathrel{R_\nat}}\M$
      if $\M$ reduces to~$\underline{n}$
      whenever $d$ is a natural number:
      \[
        d \alert{\mathrel{R_\nat}} \M
        \quad\text{if}\quad \forall n \in \mathbb{N}.\,
        d = n \implies \M \Downarrow \underline{n}
      \]
    \item for every function type, $f \alert{\mathrel{R_{\sigma\to\tau}}} \M$
      if the outcome is always related whenever the input is related:
      \[
        f \alert{\mathrel{R_{\sigma\to\tau}}} \M
        \quad\text{if}
      \]
      \[
        \forall d, 
        \N.\, 
        d \alert{\mathrel{R}_\sigma} \N \implies f(d) \alert{\mathrel{R_\tau}} \M\;\N
      \]
  \end{itemize}
\end{definition}
For example, $0 \alert{\mathrel{R_\nat}} \zero$, and
$n + 1 \alert{\mathrel{R_\nat}}
\suc\;\M$ wherever $n \alert{\mathrel{R_\nat}} \M$ for~$n \in \mathbb{N}$. 

\end{frame}
\begin{frame}{Properties of~$R_\sigma$}
\begin{lemma}
  \label{lem:property_logical_relation}
  For every type~$\sigma$, the following statements are true:
  \begin{enumerate}
    \item If $d' \sqsubseteq d$ and $d\alert{\mathrel{R_\sigma}}\M$, then
      $d' \alert{\mathrel{R_\sigma}} \M$; 
    \item For every $\M \in \mathsf{Prg}_\sigma$, the set
      \[
        R_\sigma \M \defeq \set{ d \in D_\sigma}{ d \alert{\mathrel{R_\sigma}}
          \M}
      \]
      contains~$\bot$ and is closed under directed sups;\footnote{
        Let $S$ be an arbitrary directed subset of~$D_\sigma$, 
        if $d \alert{\mathrel{R_\sigma}} \M$ for every~$d \in S$, then 
        $\bigsqcup S \alert{\mathrel{R_\sigma}} \M$. 
      }
    \item If $d \alert{\mathrel{R_\sigma}} \M$ and $M \precsim_\sigma \M'$, then
      $d \alert{\mathrel{R_\sigma}} \M'$. 
  \end{enumerate}
\end{lemma}
\begin{proof}
  By induction on~$\sigma$.
\end{proof}
\end{frame}

\begin{frame}
  \begin{lemma}[General recursion]
    If we have $f \alert{\mathrel{R_{\sigma\to\sigma}}} (\lambda x.\, \M)$, then
    $\mu(f) \alert{\mathrel{R_\sigma}} (\fix x.\, \M)$. 
  \end{lemma}
\begin{proof}[Proof sketch.]
  By definition~$\mu(f)$ is the directed supremum  of
  the following directed sequence 
  \[
    \bot \sqsubseteq f(\bot) \sqsubseteq f^2(\bot)
    \sqsubseteq \cdots \sqsubseteq f^i(\bot) \sqsubseteq
    \cdots, 
  \]
  so it suffices to show that
  \[
    f^i(\bot) \alert{\mathrel{R_\sigma}} (\fix x.\, \M)
  \]
  for every $i \in \mathbb{N}$, because $R_\sigma(\fix x.\M)$ is closed under
  directed sups. We prove it by induction on~$i$ and properties of~$R_\sigma$.
\end{proof}
\end{frame}

\begin{frame}
  The complete proof is listed below.
  \begin{description}
    \item[For $i = 0$:] By definition $f^0(\bot) = \bot$, so
      $\bot \alert{\mathrel{R_\sigma}} (\fix x.\, \M)$ follows.
      \\~\\
    \item[For $i = n + 1$:]
      By the assumption~$f\alert{\mathrel{R_{\sigma\to\sigma}}}(\lambda x.\M)$, it
      follows that
      \[
        f^{n+1}(\bot)
        \alert{\mathrel{R_\sigma}} (\lambda x.\,\M)\; (\fix x.\,\M)
      \]
      by the induction hypothesis
      $f^n(\bot) \alert{\mathrel{R_\sigma}} (\fix x.\, \M)$. 

      The RHS reduces to~$\M[\fix x.\M /x]$ and $\fix x.\M \leadsto
      \M[\fix x.\M / x]$, so the RHS is logically equivalent to~$\fix x.\M$.
      Hence, it follows that
      \[
        f^{n+1}(\bot) \alert{\mathrel{R_\sigma}} (\fix x.\,\M).
      \]
  \end{description}
  ~\\
  Therefore, it follows that $\bigsqcup_{i \in \mathbb{N}} f^{i}(\bot)
  \alert{\mathrel{R_\sigma}} (\fix x.\, \M)$. 
\end{frame}
\begin{frame}{Substitution Lemm and completeness}
  \begin{lemma}[Substitution]
    Let $\Gamma = x_1:\sigma_1, \dots, x_k:\sigma_k$ be a context and $d_i
    \alert{\mathrel{R_{\sigma_i}}} \N_i$ for $i = 1, \dots, n$.
    For every well-typed term~$\M$ we have
    \[
      \sem{\, \Gamma \vdash \M : \tau\,}(\vec{d})
      \alert{\mathrel{R_\tau}}
      \M[\vec{N}/\vec{x}]
    \]
  \end{lemma}
  \begin{theorem}[Completeness]
    For every~${}\vdash \M:\nat$, we have $\M \Downarrow \underline{n}$
    if~$\sem{\M} = n$. 
  \end{theorem}
  \begin{proof}
    A special case of the previous lemma:
    \[
      \sem{{}\vdash \M:\tau}(*) \alert{\mathrel{R_\sigma}} \M
    \]
    where the LHS is~$\sem{\M}$.
  \end{proof}
\end{frame}

\begin{frame}{Proof of Substitution Lemma}
  To prove the lemma, do induction on the typing rules for~\PCF{}. 
  For convenience, we write
  \\~\\
  \[
    \vec{d} \alert{\mathrel{R}} \vec{\N}
    \quad\text{for}\quad
    d_i \alert{\mathrel{R_{\sigma_i}}} \N_i
    \quad\text{indexed by~$i = 1, \dots, n$}
  \]
  ~\\
  where $\vec{d}$ stands for $(d_1, \dots, d_n)$
  and $\vec{\N}$ stands for $(\N_1, \dots, \N_n)$.
  \\~\\
  \begin{description}
    \item[(z), (s)] These two cases follow from
      $0\alert{\mathrel{R}_\nat}\zero$ and $n+1 \alert{\mathrel{R_\nat}}
      \suc\;\M$ whenever $n \alert{\mathrel{R_\nat}} \M$. 
  \end{description}
\end{frame}

\begin{frame}
  \begin{description}
    \item[(var)] To show that
      \[
        \sem{\dots, x_i : \sigma_i, \dots \vdash x_i : \sigma_i}
        \alert{\mathrel{R_{\sigma_i}}}
        x_i[\vec{\N}/\vec{x}]
      \]
      we check both sides separately.
      By definition, we have
      \[
        \sem{\dots, x_i : \sigma_i, \dots \vdash x_i : \sigma_i}(\vec{d})
        = d_i
        \quad\text{and}\quad
        [\vec{\N}/\vec{x}] = \N_i.  
      \]
      Therefore, from the assumption it follows
      that $d_i \alert{\mathrel{R_{\sigma}}} \N_i$ for every~$i$.
  \end{description}
\end{frame}
\begin{frame}
  \begin{description}
    \item[(abs)] We need to show that
      \begin{equation}
        \label{eq:subst_abs}
        \sem{\Gamma\vdash\lambda x.\,\M : \tau} (\vec{d})
        \alert{\mathrel{R_{\sigma\to\tau}}}
          (\lambda x.\, \M)[\vec{\N}/\vec{x}]
      \end{equation}
      under the induction hypothesis
      \[\sem{\Gamma, x : \sigma \vdash \M : \tau} (\vec{d}, d)
        \alert{\mathrel{R_{\tau}}}
          \M[\,\vec{\N}, \N\mathrel{/}\vec{x}, x\,].
      \]
      \begin{itemize}
        \item For the LHS, we have by definition
          \begin{align*}
            & \sem{\Gamma \vdash \lambda x.\M : \tau}(\vec{d})(d) \\
            ={}& \sem{\Gamma, x : \sigma \vdash \M : \tau}(\vec{d}, d).
          \end{align*}
        \item For the RHS, we have
          \begin{align*}
            & (\lambda x.\,\M)[\vec{\N}/\vec{x}]\;\N \\
            \leadsto{} & (\lambda x.\,\M)[\vec{\N}/\vec{x}][\N/x] \\
            ={} & (\lambda x.\,\M)[\, \vec{\N}, \N \mathrel{/} \vec{x}, x\,]
          \end{align*}
          and it follows that these two terms are logically equivalent. 
          Thus, \eqref{eq:subst_abs} follows by
          the definition of~$\mathrel{R_{\sigma\to\tau}}$.
      \end{itemize}
  \end{description}

\end{frame}

\begin{frame}
  \begin{description}
    \item[(Y)] We show that
      $\sem{\Gamma \vdash \fix x.\, \M : \sigma}(\vec{d})
        \alert{\mathrel{R_\sigma}}
        (\fix x.\,\M)[\vec{\N}/\vec{x}]$
      under the assumption that
      \begin{equation}
        \label{eq:subst_fix}
        \sem{\Gamma, x : \sigma \vdash \M : \sigma}(\vec{d}, d)
        \alert{\mathrel{R_\sigma}}
        \M[\,\vec{\N}, \N/\vec{x}, x\,]
      \end{equation}
      Recall the lemma for general recursion. It suffices to show
      $\Lambda\sem{\Gamma, x : \sigma \vdash \M : \sigma}(\vec{d})
      \alert{\mathrel{R_{\sigma\to\sigma}}}
        \lambda x.\, \M[\vec{\N}/ \vec{x}] $
      or, equivalently 
      \begin{equation}
        \label{eq:subst_fix2}
        \sem{\Gamma, x : \sigma \vdash \M : \sigma}(\vec{d}, d)
        \alert{\mathrel{R_{\sigma}}}
        (\lambda x.\, \M[\vec{\N}/\vec{x}])\;\N
      \end{equation}
      for every $d \mathrel{R}_\sigma \N$.
      The RHS can be reduced to
      \[
        \M[\vec{\N}/\vec{x}][\N/x]
        = \M[\,\vec{\N}, \N\mathrel{/}\vec{x}, x\,],
      \]
      so \eqref{eq:subst_fix} implies 
      \eqref{eq:subst_fix2} by logical equivalence.
    \item[(app), (ifz)] Exercises. 
  \end{description}
\end{frame}

\subsection{Applications of adequacy}

\begin{frame}{Applicative approximation coincides with logical relation}
  \begin{lemma}
    For every ${}\vdash \M:\sigma$ and ${}\vdash \N : \sigma$,
    \[
      \M \precsim_\sigma \N
      \quad\text{if and only if}\quad
      \sem{\M} \alert{\mathrel{R_\sigma}} \N.
    \]
  \end{lemma}
  \begin{proof}
    \begin{description}
      \item[$\M \precsim_\sigma \N$.] By adequacy, we have
        $\sem{\M} \alert{\mathrel{R_\sigma}} \M$, so $\sem{\M}
        \alert{\mathrel{R_\sigma}} \N$. 
      \item[$\sem{M}\alert{\mathrel{R_\sigma}} \N$.]
        Prove it by induction on~$\sigma$.
        \begin{description}[xxxxx]
          \item[$\nat$:] If $\sem{\M} \alert{\mathrel{R_\nat}} \N$, then $\N
            \Downarrow \underline{n}$ whenever $\sem{\M} = n$.
          \item[$\sigma\to\tau$:] For $\sigma\to\tau$, by adequacy, we have
            $\sem{P}\alert{\mathrel{R_\sigma}} P$ for every~$P$, so by
            assumption and $\sem{\M\;P} = \sem{\M}(\sem{P})
            \alert{\mathrel{R_\tau}} \N\;P$.  By induction hypothesis, $\M\;P
            \precsim_\tau \N\;P$ for every~$P$, so $\M
            \precsim_{\sigma\to\tau} \N$ by definition.
        \end{description}
    \end{description}
  \end{proof}
\end{frame}

\begin{frame}
  \begin{corollary}
    Given two ${}\vdash \M: \sigma$ and~${}\vdash \N : \sigma$,
    if $\sem{\M} = \sem{\N}$, then $\M$ and~$\N$ are logically equivalent. 
  \end{corollary}
  ~\\

  \begin{proof}
    \begin{enumerate}
      \item By adequacy $\sem{\M} \alert{\mathrel{R}} \M$ and by assumption
        $\sem{\N} = \sem{\M} \alert{\mathrel{R}} \M$, it follows that $\N
        \precsim \M$.  \item Similarly, 
        $\sem{\M} \alert{\mathrel{R}} \N$, so $\M \precsim \N$. 
    \end{enumerate}
    Hence, $\M$ and $\N$ are logically equivalent.
  \end{proof}
  ~\\
  From this property, techniques and results in denotational semantics can be
  used to argue logical equivalence and reductions. 
\end{frame}

\begin{frame}{Compactness}
  Recall that the semantics of general recursion is the least upper bound of
  its finite unfoldings 
  \[
    \sem{\,\fix x.\, \M\,}
    = \bigsqcup_{i \in \mathbb{N}} \sem{\,\fix^i x.\, \M\,}
  \]
  where $\fix^i x.\, \M$ is defined inductively by
  \begin{enumerate}
    \item $\fix^0 x.\, \M \defeq \fix x.\, x$ and
    \item $\fix^{n+1} x.\, \M \defeq \M[\fix^{n} x.\, \M /x]$
  \end{enumerate}
  and $\sem{\fix^i x.\,\M} = \sem{\lambda x.\, \M}^i(\bot)$.
  \begin{theorem}
    Suppose that $x \neq y$, 
    \[
      y : \sigma \vdash E : \nat
      \quad\text{and}\quad
      \vdash \fix x.\, \M : \sigma.
    \]
    If $E[ \fix x.\, \M / y] \Downarrow \underline{n}$ then
    $E[ \fix^m x.\, \M / y] \Downarrow \underline{n}$ for some~$m$.
  \end{theorem}
\end{frame}

\begin{frame}
  \begin{proof}
    By the Substitution Lemma, we have
    \[
      \sem{E[\fix x.\, \M / y]} = \sem{y : \sigma \vdash E : \nat}(\sem{\fix
        x.\, \M}).
    \]
    Let $g \defeq \sem{y : \sigma \vdash E : \nat}$
    and $f \defeq \sem{x : \sigma \vdash \M : \sigma}$. 
    \begin{align*}
      \sem{y : \sigma \vdash E : \nat}(\sem{\fix x.\,\M}) & = g(\mu f) \\
      & = g(\bigsqcup_{i \in \mathbb{N}} f^i (\bot)) \\ 
      & = \bigsqcup_{i \in \mathbb{N}} (g \circ f^i)(\bot) = n
    \end{align*}
    Therefore there exists some~$m \in \mathbb{N}$ such that $(g \circ
    f^m)(\bot) = n$. By adequacy, it follows that $E[\fix^m x.\, \M/ y]
    \Downarrow \underline{n}$.
  \end{proof}
\end{frame}

\begin{frame}{Finite unfoldings approximate general recursion} 
  \begin{lemma}
    Suppose that~$x : \sigma \vdash \M : \sigma$. Then for every~$i \in
    \mathbb{N}$, we have
    \[
      \fix^i x.\,\M \precsim_\sigma \fix x.\,\M. 
    \]
  \end{lemma}
  The proof is left as an exercise. 
  \begin{theorem}[Fixed Point Induction]
    Suppose that $x : \sigma \vdash \M : \sigma$, $x : \sigma \vdash \N
    :\sigma$ and 
    \[
      \fix^i x.\,\M \simeq_\sigma \fix^i x.\,\N
    \]
    for every~$i \in \mathbb{N}$.
    Then, we also have
    \[
      \fix x.\,\M
      \mathrel{\simeq_\sigma}
      \fix x.\, \N
    \]
  \end{theorem}
\end{frame}

\begin{frame}
  \begin{proof}
    We show that $\fix x.\, \M \precsim_\sigma \fix x.\, \N$, or
    equivalently~$\sem{\fix x.\, \M} \alert{\mathrel{R_\sigma}} \fix x.\, \N$,
    and the other direction follows similarly. 
    \\~\\
    
    Let $f \defeq \sem{x : \sigma \vdash \M : \sigma}$ and $g \defeq \sem{x :
      \sigma \vdash \N : \sigma}$. Since the set
    \[
      R_\sigma(\fix x.\,\N)
      = \set{ d \in D_\sigma}{ d \alert{\mathrel{R_\sigma}} \fix x.\,\N}
    \]
    is closed under directed supremum, it suffices to show that
    \[
      \sem{\fix^i x.\,\M} \alert{\mathrel{R_\sigma}} \fix x.\,\N
    \]
    for every~$i$.
    \\~\\
    
    By assumption, we have $\sem{\fix^i x.\,\M}\alert{\mathrel{R_\sigma}}
    \fix^i x.\,\N$, so it suffices to show that $\fix^i x.\,\N \precsim_\sigma
    \fix x.\,\N$. By the previous lemma the statement follows. 
  \end{proof}
\end{frame}

\begin{frame}{Exercises}
  Show that the following pairs of programs are logically equivalent.
  \begin{enumerate}
    \item 
  \end{enumerate}
\end{frame}
\end{document}
