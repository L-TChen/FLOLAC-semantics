\title{Semantics of Functional Programming}
\subtitle{Lecture IV: Computational Adequacy and Further Topics}
\author[L.-T. Chen]{Chen, Liang-Ting\\
  \href{mailto:lxc@iis.sinica.edu.tw}{\texttt{lxc@iis.sinica.edu.tw}}}
\institute[IIS, Sinica]{Institute of Information Science, Academia Sinica}
\begin{document}
\frame{\maketitle}

\begin{frame}
  So far we have given two kinds of semantics for \PCF{}. For a program~$\M$ of
  type~$\sigma$,
  \begin{itemize}
    \item one gives how the program~$\M$ is evaluated to a closed value~$\V$
      via the reduction relation~$\M \Downarrow\V$;
    \item the other defines what the denotation~$\sem{\M}$ of~$\M$ is.
    \end{itemize}
  In this lecture, we will compare these two approaches and discuss some issues
  arising from them:
  \begin{description}[Computational Adequacy]
    \item[Correctness]
    \item[Completeness]
    \item[Computational adequacy]
    \item[Full abstraction]
  \end{description}

\end{frame}
\section{Computational Adequacy}  
\begin{frame}{Closed values of $\nat$ do not diverge}
  The bottom element~$\bot$ in a domain models the divergence of computation,
  and a closed value~$\V$ of~$\nat$ is meant to be some natural
  number. Let's justify this idea.
  \begin{lemma}
    For every closed value~$\V$ of type~$\nat$, the denotation~$\sem{\V}$ is an
    element of~$\mathbb{N}$. In particular, $\sem{\V} \neq \bot$.
  \end{lemma}
  \begin{proof}
    By structural induction on closed values. 
    For the following cases
    \begin{columns}[b]
      \column{.3\textwidth}
        \begin{prooftree}
          \AXC{}
          \UIC{$\zero\;\,\val$}
        \end{prooftree}
      \column{.3\textwidth}
        \begin{prooftree}
          \AXC{$\M\;\,\val$}
          \UIC{$\suc\;\M\;\,\val$}
        \end{prooftree}
      \column{.3\textwidth}
        \begin{prooftree}
          \AXC{}
          \UIC{$\lambda x.\, \M\;\,\val$}
        \end{prooftree}
      \end{columns}
    it is easy to see that $\sem{\zero}$ and $\sem{\suc\;\M}$, if $\sem{\M} \in
    \mathbb{N}$, are elements of~$\mathbb{N}$ by the definition
    of~$\sem{-}$. On the other hand, $\lambda x.\, \M$ cannot be of
    type~$\nat$, so this case holds vacuously.
  \end{proof}
  By inspection of the above proof, we conclude that $\sem{\underline{n}} = n$
  --- what a numeral~$\underline{n}$ of~$n$ should mean. 
\end{frame}
\begin{frame}{Correctness}
  Now we show that denotational semantics is correct with respect to
  denotational semantics:
  \begin{theorem}
    For every two programs~$\M$ and~$\V$, $\M \Downarrow \V$ implies
    $\sem{\M} = \sem{\V}$. 
  \end{theorem}

  A sanity check: By Preservation Theorem, it is known that $\sem{{}\vdash \M :
    \tau}$ and $\sem{{}\vdash \V : \sigma}$ are of the same type if $\M
  \Downarrow \V$, so their range~$\sem{\tau}$ and $\sem{\sigma}$ are the same.

  \begin{proof}[Proof Sketch]
    Prove $\sem{\M} = \sem{\V} \in \sem{\tau}$ by structural induction on the
    derivation of~$\M\Downarrow\V$. 
  \end{proof}
\end{frame}

\begin{frame}{Proof of correctness}
    We show the case ($\Downarrow$-suc) first and the cases
    ($\Downarrow$-zero) and ($\Downarrow$-lam) are similar and easy. 
    \begin{itemize}
      \item For ($\Downarrow$-suc), we show that $\sem{\suc\;\M}
        =\sem{\suc\;\V}$ if $\sem{\M} = \sem{\V}$. By definition, we simply
        calculate its denotation directly:
        \[
          \sem{\suc\;\M} = S(\sem{\M}) = S(\sem{\V}) = \sem{\suc\;\V}
        \]
        where the middle equality follows from the induction hypothesis. 
    \end{itemize}
    Try to do the cases ($\Downarrow$-zero), ($\Downarrow$-lam), and
    ($\Downarrow$-$\ifz_0$).
\end{frame}

\begin{frame}
  The case ($\Downarrow$-app) is slightly complicated, as we have to address
  the binding structure using Substitution Lemma.  
  \begin{itemize}
      \item For ($\Downarrow$-app), we show that $\sem{\M\;\N} = \sem{V}$
        if $\sem{\M} = \sem{\lambda x.\, \mathsf{E}}$ and $\sem{E[\N/x]} =
        \sem{V}$. We calculate the denotation as follows
        \begin{align*}
          \sem{\M\;\N} & = \mathit{ev}(\sem{\M}, \sem{\N}) \\
          & = \mathit{ev}(\sem{\lambda x.\, \mathsf{E}}, 
          \sem{\N}) \\
          & = \mathit{ev}(\sem{x : \sigma \vdash \mathsf{E} : \tau},
          \sem{\N}) \\
          & = \sem{x : \sigma \vdash \mathsf{E} : \tau}(\sem{\N})
          = \sem{\mathsf{E}[\N/x]} = \sem{\V}
        \end{align*}
        where the last but one equation follows from Substitution Lemma.
      \item Complete the remaining two (interesting) cases
        ($\Downarrow$-$\ifz_1$) and ($\Downarrow$-fix).
        \emph{Hint.} Consider Substitution Lemma and
        the properties of the fixpoint operator~$\mu$.
    \end{itemize}
\end{frame}

\note{
\begin{itemize}
  \item For ($\Downarrow$-$\ifz_0$), assuming $\sem{\M}=\sem{\zero} = 0$
    and $\sem{\M_0} = \sem{V}$ we show that
    $\sem{\ifz(\M; \M_0; x.\, \M_1)} = \sem{\V}$. We calculate the
    denotation as follows
    \begin{align*}
      \sem{\ifz(\M;\M_0;x.\,\M_1)}
        & = \mathit{ifz}(\sem{\M}, \sem{\M_0}, \sem{\M_1}) \\
        & = \mathit{ifz}(0, \sem{V}, \sem{\M_1}) \\
        & = \sem{V}
    \end{align*}
    where the last equation follows from the definition of~$\mathit{ifz}$. 
  \item For ($\Downarrow$-$\ifz_1$), we show that $\sem{\ifz(\M;\M_0;
      x.\,\M_1)}$ = \sem{V} if $\sem{\M} = \sem{\suc\;\N} = S \circ \sem{\N}$
    and $\sem{\M_1[\N/x]} = \sem{\V}$.
  
  We know that~$\V$ is a closed value for every~$\M\Downarrow\V$ and $\sem{\V}
  = n$ for some natural number~$n \in \mathbb{N}$, so $\sem{\M}$ is a natural
  number $n + 1$. It follows that 
  \begin{align*}
    \sem{\ifz(\M; \M_0; x.\,\M_1)}
    & = \mathit{ifz}(\sem{\M}, \sem{\M_0}, \sem{x : \nat\vdash \M_1 :
      \tau}) \\
    & = \sem{x : \nat \vdash \M_1}(n) \\
    & = \sem{x : \nat \vdash \M_1}(\sem{N}) \\
    & = \sem{\M_1[\N/x]} = \sem{\V}
  \end{align*}
  where the last but one equation follows from Substitution Lemma.

  \item For ($\Downarrow$-fix), we show that $\sem{\fix x.\,\M} = \sem{\V}$ if
    $\sem{\M[\fix x.\,\M/x]} = \sem{\V}$.  Let $f \defeq \sem{x : \sigma \vdash
      \M : \sigma}$.  We
    calculate the denotation as follows
    \begin{align*}
      \sem{\fix x.\,\M}
      & = \mu f = f(\mu f) \\
      & = \sem{x : \sigma \vdash \M : \sigma}(\sem{\fix x.\, \M}) \\
      & = \sem{\M[\fix x.\, \M/x]} = \sem{V}
    \end{align*}
    where the last but one equality follows from Substitution Lemma.
\end{itemize}
}
\end{document}
